%%%%%%%%%%%%%%%%%%%%%%%%%%%%%%%%%%%%%%%%
% Classe do documento
%%%%%%%%%%%%%%%%%%%%%%%%%%%%%%%%%%%%%%%%

% Nós usamos a classe "unb-cic".  Deixe apenas uma das linhas
% abaixo não-comentada, dependendo se você for do bacharelado ou
% da licenciatura.

\documentclass[bacharelado]{unb-cic}
%\documentclass[licenciatura]{unb-cic}



%%%%%%%%%%%%%%%%%%%%%%%%%%%%%%%%%%%%%%%%
% Pacotes importados
%%%%%%%%%%%%%%%%%%%%%%%%%%%%%%%%%%%%%%%%

\usepackage[brazil,american]{babel}
\usepackage[T1]{fontenc}
\usepackage{indentfirst}
\usepackage{natbib}
\usepackage{xcolor,graphicx,url}
\usepackage[utf8]{inputenc}



%%%%%%%%%%%%%%%%%%%%%%%%%%%%%%%%%%%%%%%%
% Cores dos links
%%%%%%%%%%%%%%%%%%%%%%%%%%%%%%%%%%%%%%%%

% Veja o arquivos cores.tex se quiser ver que outras cores estão
% pré-definidas.  Utilizando o comando \hypersetup abaixo nós
% evitamos aquelas caixas vermelhas feias em volta dos links.

%%%%%%%%%%%%%%%%%%%%%%%%%%%%%%%%%%%%%%%%
% Cores do estilo Tango
%%%%%%%%%%%%%%%%%%%%%%%%%%%%%%%%%%%%%%%%

\definecolor{LightButter}{rgb}{0.98,0.91,0.31}
\definecolor{LightOrange}{rgb}{0.98,0.68,0.24}
\definecolor{LightChocolate}{rgb}{0.91,0.72,0.43}
\definecolor{LightChameleon}{rgb}{0.54,0.88,0.20}
\definecolor{LightSkyBlue}{rgb}{0.45,0.62,0.81}
\definecolor{LightPlum}{rgb}{0.68,0.50,0.66}
\definecolor{LightScarletRed}{rgb}{0.93,0.16,0.16}
\definecolor{Butter}{rgb}{0.93,0.86,0.25}
\definecolor{Orange}{rgb}{0.96,0.47,0.00}
\definecolor{Chocolate}{rgb}{0.75,0.49,0.07}
\definecolor{Chameleon}{rgb}{0.45,0.82,0.09}
\definecolor{SkyBlue}{rgb}{0.20,0.39,0.64}
\definecolor{Plum}{rgb}{0.46,0.31,0.48}
\definecolor{ScarletRed}{rgb}{0.80,0.00,0.00}
\definecolor{DarkButter}{rgb}{0.77,0.62,0.00}
\definecolor{DarkOrange}{rgb}{0.80,0.36,0.00}
\definecolor{DarkChocolate}{rgb}{0.56,0.35,0.01}
\definecolor{DarkChameleon}{rgb}{0.30,0.60,0.02}
\definecolor{DarkSkyBlue}{rgb}{0.12,0.29,0.53}
\definecolor{DarkPlum}{rgb}{0.36,0.21,0.40}
\definecolor{DarkScarletRed}{rgb}{0.64,0.00,0.00}
\definecolor{Aluminium1}{rgb}{0.93,0.93,0.92}
\definecolor{Aluminium2}{rgb}{0.82,0.84,0.81}
\definecolor{Aluminium3}{rgb}{0.73,0.74,0.71}
\definecolor{Aluminium4}{rgb}{0.53,0.54,0.52}
\definecolor{Aluminium5}{rgb}{0.33,0.34,0.32}
\definecolor{Aluminium6}{rgb}{0.18,0.20,0.21}

\hypersetup{
  colorlinks=true,
  linkcolor=DarkScarletRed,
  citecolor=DarkScarletRed,
  filecolor=DarkScarletRed,
  urlcolor= DarkScarletRed
}



%%%%%%%%%%%%%%%%%%%%%%%%%%%%%%%%%%%%%%%%
% Informações sobre a monografia
%%%%%%%%%%%%%%%%%%%%%%%%%%%%%%%%%%%%%%%%

\title{Feature Featherweight Java: Implementação e Formalização}

\orientador{\prof \dr Rodrigo Bonifácio de Almeida}{CIC/UnB}
%\coorientador[a]{\prof[a] \dr[a] Coorientadora}{MAT/UnB}
\coordenador{\prof \dr Flávio Vidal}{CIC/UnB}
\diamesano{11}{abril}{2016}

\membrobanca{\prof \dr Rodrigo B. de Almeida}{CIC/UnB}
\membrobanca{\prof \dr Flávio L. C Moura}{CIC/UnB}

\autor{Pedro da C.}{Abreu Jr.}
\coautor{Daniella A.}{dos Angelos}
\CDU{004.4}

\palavraschave{palvrachave1, palvrachave2, palvrachave3 }
\keywords{keyword1, keyword2, keyword3}



%%%%%%%%%%%%%%%%%%%%%%%%%%%%%%%%%%%%%%%%
% Texto
%%%%%%%%%%%%%%%%%%%%%%%%%%%%%%%%%%%%%%%%

\begin{document}
  \maketitle
  \pretextual

%  \begin{dedicatoria}
%  Dedico a \dots
%  \end{dedicatoria}%

%  \begin{agradecimentos}
%  Agradeço a \dots
%  \end{agradecimentos}

  \begin{resumo}
  A ciência \dots
  \end{resumo}

  \selectlanguage{american}
  \begin{abstract}
  The science \dots
  \end{abstract}
  \selectlanguage{brazil}

  \tableofcontents
  \listoffigures
  \listoftables

  \textual
  
  \chapter{Introdução}

	Com o advento da teoria das provas e da teoria de tipos, ferramentas robustas para prova de corretude de linguagens de computação vêm sendo introduzidas e melhoradas nestas últimas duas décadas. Entre eles podemos citar SAT solvers, Provadores de teoremas automatizados e provadores de teoremas assistidos.
	Neste artigo estamos interessados em provadores de teoremas assistidos, mais especificamente Coq. 
	Com o Coq iremos implementar a especificação de o core funcional da linguagem Java chamado Featherweight Java \cite{Igarashi99featherweightjava:} provando sua corretude (Type Safety) e mais tarde expandir sua espeficação com Features, especificado em \cite{Apel08featurefeatherweight}. Como ainda não existe uma formalização de Feature Featherweight Java assistida por máquina até a data atual, este trabalho é uma novidade no meio academico.
	
	\section{Coq}
	Coq é um provador de teoremas automatizado  baseado na linguagem formal Calculo de Construções Indutivas, o qual em si mesmo combina tanto lógica de ordem superior quanto uma linguagem funcional ricamente-tipada. Através de uma linguagem vernacular de comandos, Coq permite:
	\begin{itemize}
		\item definir funções ou predicados, que podem ser avaliados eficientemente;
		\item declarar teoremas matemáticos e especificações de software;
		\item desenvolver provas formas destes teoremas de forma iterativa;
		\item checar estas provas por uma certificação de kernell relativamente pequena;
		\item extrair programas certificados para Objective Caml, Haskell ou Scheme.
	\end{itemize}
	Como um sistema de desenvolvimento de provas, Coq provê métodos de provas iterativos, algoritmos de decisão e semi-decisão, e uma linguagem de tatica que permite o usuário definir seus próprios métodos de prova. Conexão com uma calculadora de sistema algébrico externo e outros provadores de teoremas está disponível.
	Como uma plataforma de formalização matemática ou de desenvolvimento de programas, coq provê suporte para notações de alta ordem, conteúdos implicitos e vários outros tipos úteis de macros.








  %\chapter{Implementation details}

\section{Why Haskell?}

\section{BNFC}

\section{Monads}

\section{Coq} 
Coq is a formal proof management system based on the Calculus of Inductive
Constructions~\cite{coqart}. It provides a formal language to write mathematical
definitions, executable algorithms and theorems together with an environment
that combines superior order logic and a dependently typed functional language.
Coq allows you to, with this language~\cite{coqsite}:
\begin{itemize}
\item define functions or predicates, that can be efficiently evaluated
\item declare mathemematical theorems and software specifcations
\item develop formal proofs
\item check these proofs by a certificate ``kernel''
\item extract certified programs to some functional programming languages
\end{itemize}


\iffalse
Coq é um provador de teoremas automatizado baseado na
linguagem formal Calculo de Construções Indutivas\cite{coqart}, o qual em si
mesmo combina tanto lógica de ordem superior quanto uma linguagem funtional
ricamente-tipada. Através de uma linguagem vernacular de comandos, Coq
permite: \begin{itemize} \item definir funções ou predicados, que podem ser
avaliados eficientemente; \item declarar teoremas matemáticos e
especificações de software; \item desenvolver provas formas destes teoremas
de forma iterativa; \item checar estas provas por uma certificação de
kernel relativamente pequena; \item extrair programas certificados para
Objective Caml, Haskell ou Scheme.  \end{itemize} Como um sistema de
desenvolvimento de provas, Coq provê métodos de provas iterativos,
algoritmos de decisão e semi-decisão, e uma linguagem de tatica que permite
o usuário definir seus próprios métodos de prova. Conexão com uma
calculadora de sistema algébrico externo e outros provadores de teoremas
está disponível.  Como uma plataforma de formalização matemática ou de
desenvolvimento de programas, coq provê suporte para notações de alta ordem,
conteúdos implicitos e vários outros tipos úteis de macros.
\fi











  % ...

  \postextual
  \bibliographystyle{plain}
  \bibliography{bibliografia}

\end{document}
