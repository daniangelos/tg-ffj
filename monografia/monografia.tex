%%%%%%%%%%%%%%%%%%%%%%%%%%%%%%%%%%%%%%%%
% Classe do documento
%%%%%%%%%%%%%%%%%%%%%%%%%%%%%%%%%%%%%%%%

% Nós usamos a classe "unb-cic".  Deixe apenas uma das linhas
% abaixo não-comentada, dependendo se você for do bacharelado ou
% da licenciatura.

\documentclass[bacharelado]{unb-cic}
%\documentclass[licenciatura]{unb-cic}



%%%%%%%%%%%%%%%%%%%%%%%%%%%%%%%%%%%%%%%%
% Pacotes importados
%%%%%%%%%%%%%%%%%%%%%%%%%%%%%%%%%%%%%%%%

\usepackage[brazil,american]{babel}
\usepackage[T1]{fontenc}
\usepackage{indentfirst}
\usepackage{natbib}
\usepackage{mathpartir}
\usepackage{xcolor,graphicx,url}
\usepackage[utf8]{inputenc}
\usepackage{listings}



%%%%%%%%%%%%%%%%%%%%%%%%%%%%%%%%%%%%%%%%
% Cores dos links
%%%%%%%%%%%%%%%%%%%%%%%%%%%%%%%%%%%%%%%%

% Veja o arquivos cores.tex se quiser ver que outras cores estão
% pré-definidas.  Utilizando o comando \hypersetup abaixo nós
% evitamos aquelas caixas vermelhas feias em volta dos links.

\input{cores}
\hypersetup{
  colorlinks=true,
  linkcolor=DarkScarletRed,
  citecolor=DarkScarletRed,
  filecolor=DarkScarletRed,
  urlcolor= DarkScarletRed
}



%%%%%%%%%%%%%%%%%%%%%%%%%%%%%%%%%%%%%%%%
% Informações sobre a monografia
%%%%%%%%%%%%%%%%%%%%%%%%%%%%%%%%%%%%%%%%

\title{Feature Featherweight Java: Implementação e Formalização}

\orientador{\prof \dr Rodrigo Bonifácio de Almeida}{CIC/UnB}
%\coorientador[a]{\prof[a] \dr[a] Coorientadora}{MAT/UnB}
\coordenador{\prof \dr Flávio Vidal}{CIC/UnB}
\diamesano{11}{abril}{2016}

\membrobanca{\prof \dr Rodrigo B. de Almeida}{CIC/UnB}
\membrobanca{\prof \dr Flávio L. C Moura}{CIC/UnB}

\autor{Pedro da C.}{Abreu Jr.}
\coautor{Daniella A.}{dos Angelos}
\CDU{004.4}

\palavraschave{Verificação Formal, FFJ, Coq }
\keywords{Formal Verification, FFJ, Coq}



%%%%%%%%%%%%%%%%%%%%%%%%%%%%%%%%%%%%%%%%
% Texto
%%%%%%%%%%%%%%%%%%%%%%%%%%%%%%%%%%%%%%%%

\begin{document}
  \maketitle
  \pretextual

 % \begin{dedicatoria}
  %Dedico a \dots
  %\end{dedicatoria}

  %\begin{agradecimentos}
  %Agradeço a \dots
  %\end{agradecimentos}

  %\begin{resumo}
  %Especificação formal de linguagens vem provado sua eficácia na detecção de possíveis bugs e principalmente no entendimento mais profundo na estrutura da linguagem. Ferramentas de especificação formal e em provas assistida por computadores, como por exemplo Coq, vem sendo alvo de interesse crescente devido à robustez teórica. O objetivo primordial deste trabalho é juntar estas duas frentes e especificar formalmente Feature Featherweight Java em Coq e provar sua corretude.
  %\end{resumo}

  \selectlanguage{american}
  \begin{abstract}
  Languages formal especification are very efective to detect bugs and more importantly provides a deeper understanding of the structure of such. Formal especification tools, such as Coq, have been target of great research interest in the previous years, one of the main reasons is for its proved soundnes. The main objective of this paper is to bring together these two front and specify formaly Feature Featherweight Java with Coq and prove its soundnes.
  \end{abstract}
  \selectlanguage{brazil}

  \tableofcontents
  %\listoffigures
  %\listoftables

  \textual
  \chapter{Introdução}

\section{Formal Semantics} In this article we are concearned in studying the
formal semantics of programming languages, i.e.\ we are interested in the tools
to understand and reason how programs behave. More specifically the semantics of
Feature Featherweight Java.  For historical reasons the semantics of programming
languages is often viewed as consisteing of three strands: 

\begin{itemize} 
    \item \textit{Operational semantics} describes the meaning of the
        programming language by specifying how it executes on an abstract machine;

    \item \textit{Denotational semantics} is a technique for defining the
    meaning of programming languages. At one time called ``mathematical
    semantics'' it uses the more abstract mathematical concepts of complete
    partial orders, continous functions and least fixed points; 

    \item \textit{Axiomatic semantics} tries to fix the meaning of a programming
        contruct by giving proof rules for it whithin a program logic. Thus
        axiomatic semantics emphasises proof of correctness right from the start.
\end{itemize}

It would however be wrong to view these three styles as in opposition to
each other. They each have their uses. A clear operational semantics is very
helpful in implementation. Axiomatic semantics for special kinds of
languages can give strikingly elegant proof systems, useful in developing as
well as verifying programs. Denotational semantics provides the deepest and
most widely applicable thecniques, underpinned by a rich mathematical
theory. Indeed, the different styles  of semantics are highly dependent on
each other. 

For this we will need a nice understanding in logic and set theory.

Com o advento da teoria das provas e da teoria de tipos\cite{martinlof},
ferramentas robustas para prova de corretude de linguagens de computação vêm
sendo introduzidas e melhoradas nas últimas décadas. Entre eles podemos citar
SAT solvers, Provadores de teoremas automatizados e provadores de teoremas
assistidos.  Neste artigo estamos interessados em provadores de teoremas
assistidos, mais especificamente Coq, como ferramenta de auxílio na formalização
da nossa implementação da linguagem \textit{Feature Featherweigth Java} (FFJ),
introduzida nas seções seguintes. 

A formalização pode oferecer impulsos significativos no processo de abstração de
um artefato do mundo real, como, por exemplo, a abstração representada por
linguagens de programação.  Um modelo formal pode ser usado para descrever
alguns aspectos precisos do projeto, para definir e provar suas propriedades,
além de direcionar o foco a questões que poderiam vir a ser negligenciadas.
Estes são alguns dos objetivos que buscamos ao recorrer à formalização de nossa
implementação.

    \section{Coq} Coq é um provador de teoremas automatizado  baseado na
    linguagem formal Calculo de Construções Indutivas\cite{coqart}, o qual em si
    mesmo combina tanto lógica de ordem superior quanto uma linguagem funtional
    ricamente-tipada. Através de uma linguagem vernacular de comandos, Coq
    permite: \begin{itemize} \item definir funções ou predicados, que podem ser
    avaliados eficientemente; \item declarar teoremas matemáticos e
    especificações de software; \item desenvolver provas formas destes teoremas
    de forma iterativa; \item checar estas provas por uma certificação de
    kernell relativamente pequena; \item extrair programas certificados para
    Objective Caml, Haskell ou Scheme.  \end{itemize} Como um sistema de
    desenvolvimento de provas, Coq provê métodos de provas iterativos,
    algoritmos de decisão e semi-decisão, e uma linguagem de tatica que permite
    o usuário definir seus próprios métodos de prova. Conexão com uma
    calculadora de sistema algébrico externo e outros provadores de teoremas
    está disponível.  Como uma plataforma de formalização matemática ou de
    desenvolvimento de programas, coq provê suporte para notações de alta ordem,
    conteúdos implicitos e vários outros tipos úteis de macros.

    \section{Featherweight Java}

    Como FFJ é construída com base em \textit{Featherweight Java} (FJ),
    introduziremos brevemente as principais características desta linguagem.

\textit{Featherweight Java} foi proposta por Igarashi et.\ al., como um cálculo
mínimamente funcional para modelar o sistema de tipos da linguagem
Java~\cite{Igarashi99featherweightjava}.  O projeto desta linguagem favorece
compacidade ao invés de completude, possuindo apenas cinco formas de expressão:
criação de objeto, invocação de método, acesso a atributo, coerção e variáveis.
O propósito foi omitir o máximo de características da linguagem enquanto ainda
mantém o núcleo principal, para facilitar o modelo de formalização. Há uma
correspondência direta entre FJ e um núcleo puramente funcional de Java, no
sentido que todo programa em FJ é literalmente um executável de Java.


A seguir um exemplo de programa em FJ\@.

\begin{verbatim} 
class A extends Object {
    A() { super(); } 
} 

class B extends Object { 
B() { super(); }
} 

class Pair extends Object { 
    Object fst; 
    Object snd;
    Pair(Object fst, Object snd) { 
        super(); 
        this.fst = fst; 
        this.snd = snd; 
    } 
    Pair setfst(Object newfst) { 
        return new Pair(newfst, this.snd); 
    } 
} \end{verbatim}

\section{Feature Featherweight Java}

FFJ extende FJ com novas construções de linguagem orientados a composição de
\textit{features}\cite{fop}, avaliação de acordos e regras de
tipo~\cite{Apel08featurefeatherweight}. Em FFJ, um programador pode adicionar
novas classes a um programa através da introdução de uma nova característica.
    
%texto.... referência~\cite{carpenter91}









  \chapter{Implementation details}

\section{Why Haskell?}

\section{BNFC}

\section{Monads}

\section{Coq} 
Coq is a formal proof management system based on the Calculus of Inductive
Constructions~\cite{coqart}. It provides a formal language to write mathematical
definitions, executable algorithms and theorems together with an environment
that combines superior order logic and a dependently typed functional language.
Coq allows you to, with this language~\cite{coqsite}:
\begin{itemize}
\item define functions or predicates, that can be efficiently evaluated
\item declare mathemematical theorems and software specifcations
\item develop formal proofs
\item check these proofs by a certificate ``kernel''
\item extract certified programs to some functional programming languages
\end{itemize}


\iffalse
Coq é um provador de teoremas automatizado baseado na
linguagem formal Calculo de Construções Indutivas\cite{coqart}, o qual em si
mesmo combina tanto lógica de ordem superior quanto uma linguagem funtional
ricamente-tipada. Através de uma linguagem vernacular de comandos, Coq
permite: \begin{itemize} \item definir funções ou predicados, que podem ser
avaliados eficientemente; \item declarar teoremas matemáticos e
especificações de software; \item desenvolver provas formas destes teoremas
de forma iterativa; \item checar estas provas por uma certificação de
kernel relativamente pequena; \item extrair programas certificados para
Objective Caml, Haskell ou Scheme.  \end{itemize} Como um sistema de
desenvolvimento de provas, Coq provê métodos de provas iterativos,
algoritmos de decisão e semi-decisão, e uma linguagem de tatica que permite
o usuário definir seus próprios métodos de prova. Conexão com uma
calculadora de sistema algébrico externo e outros provadores de teoremas
está disponível.  Como uma plataforma de formalização matemática ou de
desenvolvimento de programas, coq provê suporte para notações de alta ordem,
conteúdos implicitos e vários outros tipos úteis de macros.
\fi











  % ...

  \postextual
  \bibliographystyle{plain}
  \bibliography{bibliografia}

\end{document}
