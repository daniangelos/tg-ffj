\chapter{Introdução}
	Com o advento da teoria das provas e da teoria de tipos, ferramentas robustas para prova de corretude de linguagens de computação vêm sendo introduzidas e melhoradas nestas últimas duas décadas. Entre eles podemos citar SAT solvers, Provadores de teoremas automatizados e provadores de teoremas assistidos.
	Neste artigo estamos interessados em provadores de teoremas assistidos, mais especificamente Coq. 
	\section{Coq}
	Coq é um provador de teoremas automatizado  baseado na linguagem formal Calculo de Construções Indutivas, o qual em si mesmo combina tanto lógica de ordem superior quanto uma linguagem funtional ricamente-tipada. Através de uma linguagem vernacular de comandos, Coq permite:
	\begin{itemize}
		\item definir funções ou predicados, que podem ser avaliados eficientemente;
		\item declarar teoremas matemáticos e especificações de software;
		\item desenvolver provas formas destes teoremas de forma iterativa;
		\item checar estas provas por uma certificação de kernell relativamente pequena;
		\item extrair programas certificados para Objective Caml, Haskell ou Scheme.
	\end{itemize}
	Como um sistema de desenvolvimento de provas, Coq provê métodos de provas iterativos, algoritmos de decisão e semi-decisão, e uma linguagem de tatica que permite o usuário definir seus próprios métodos de prova. Conexão com uma calculadora de sistema algébrico externo e outros provadores de teoremas está disponível.
	Como uma plataforma de formalização matemática ou de desenvolvimento de programas, coq provê suporte para notações de alta ordem, conteúdos implicitos e vários outros tipos úteis de macros.
	
%texto.... referência~\cite{carpenter91}








