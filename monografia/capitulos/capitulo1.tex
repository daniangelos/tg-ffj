\chapter{Introdução}
Com o advento da teoria das provas e da teoria de tipos\cite{martinlof}, ferramentas robustas para prova de corretude de linguagens de computação vêm sendo introduzidas e melhoradas nas últimas décadas. Entre eles podemos citar SAT solvers, Provadores de teoremas automatizados e provadores de teoremas assistidos.
Neste artigo estamos interessados em provadores de teoremas assistidos, mais especificamente 
Coq, como ferramenta de auxílio na formalização da nossa implementação da linguagem 
\textit{Feature Featherweigth Java} (FFJ), introduzida nas seções seguintes. 

A formalização pode oferecer impulsos significativos no processo de abstração de um artefato
do mundo real, como, por exemplo, a abstração representada por linguagens de programação. 
Um modelo formal pode ser usado 
para descrever alguns aspectos precisos do projeto, para definir e provar suas propriedades, 
além de direcionar o foco a questões que poderiam vir a ser negligenciadas. 
Estes são alguns dos objetivos que buscamos ao recorrer à formalização de nossa implementação.

	\section{Coq}
	Coq é um provador de teoremas automatizado  baseado na linguagem formal Calculo de Construções Indutivas\cite{coqart}, o qual em si mesmo combina tanto lógica de ordem superior quanto uma linguagem funtional ricamente-tipada. Através de uma linguagem vernacular de comandos, Coq permite:
	\begin{itemize}
		\item definir funções ou predicados, que podem ser avaliados eficientemente;
		\item declarar teoremas matemáticos e especificações de software;
		\item desenvolver provas formas destes teoremas de forma iterativa;
		\item checar estas provas por uma certificação de kernell relativamente pequena;
		\item extrair programas certificados para Objective Caml, Haskell ou Scheme.
	\end{itemize}
	Como um sistema de desenvolvimento de provas, Coq provê métodos de provas iterativos, algoritmos de decisão e semi-decisão, e uma linguagem de tatica que permite o usuário definir seus próprios métodos de prova. Conexão com uma calculadora de sistema algébrico externo e outros provadores de teoremas está disponível.
	Como uma plataforma de formalização matemática ou de desenvolvimento de programas, coq provê suporte para notações de alta ordem, conteúdos implicitos e vários outros tipos úteis de macros.

	\section{Featherweight Java}

	Como FFJ é construída com base em \textit{Featherweight Java} (FJ), introduziremos
brevemente as principais características desta linguagem.

\textit{Featherweight Java} foi proposta por Igarashi et.\ al., como um cálculo 
mínimamente funcional para modelar o sistema de tipos da linguagem Java~\cite{Igarashi99featherweightjava}.
O projeto desta linguagem favorece compacidade ao invés de completude, possuindo apenas 
cinco formas de expressão: criação de objeto, invocação de método, acesso a atributo, 
coerção e variáveis. O propósito foi omitir o máximo de características da linguagem 
enquanto ainda mantém o núcleo principal, para facilitar o modelo de formalização. Há 
uma correspondência direta entre FJ e um núcleo puramente funcional de Java, no sentido 
que todo programa em FJ é literalmente um executável de Java.


A seguir um exemplo de programa em FJ.

\begin{lstlisting}
class A extends Object {
	A() { super(); }
}
class B extends Object {
	B() { super(); }
}
class Pair extends Object {
	Object fst;
	Object snd;
	Pair(Object fst, Object snd) {
		super();
		this.fst = fst;
		this.snd = snd;
	}
	Pair setfst(Object newfst) {
		return new Pair(newfst, this.snd);
	}
}
\end{lstlisting}

\section{Feature Featherweight Java}

FFJ extende FJ com novas construções de linguagem orientados a composição de \textit{features}\cite{fop}, avaliação de acordos e regras de tipo~\cite{Apel08featurefeatherweight}. Em FFJ, um programador 
pode adicionar novas classes a um programa através da introdução de uma nova característica.
	
%texto.... referência~\cite{carpenter91}








