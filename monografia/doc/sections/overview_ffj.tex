\chapter{Overview of \ac{FFJ} and \ac{FFJ+}}\label{seq:offj}

\acf{FFJ} is a core calculus for \acf{FOP}, which was built 
upon an extension of \acf{FJ}---a minimal subset of Java. 
In \ac{FFJ}, classes can be added and modified by the 
introduction of a new feature, that is, 
an existing class can be extended by a class refinement. 
A class refinement is declared like a conventional class, though 
preceded by the keyword \texttt{refines}. For example, 
\texttt{refines class C \{$\dots$\}} refers to a class
refinement that
\emph{refines} the class \texttt{C}. The same can be achieved 
for method introduction and modification. Methods refinement,
however, override a previous definition of the corresponding 
method.
 
% \ac{FJ} models a minimal subset of Java. \ac{FFJ} extends \ac{FJ} for \ac{FOP}.
To fully mechanize \ac{FFJ}, we had to disambiguate and enhance 
the language to some extent that it  deserves the attention of 
formally documenting these changes. 
Even though these changes are significant, as discussed in Section~\ref{seq:ffj}, 
the philosophy of \ac{FFJ}, \ac{FOP}, and Stepwise Refinement are maintained.
Due to the lack of space we will not provide neither the formal definition 
of \ac{FJ} nor \ac{FFJ}, but we refer the reader the original formalization of \ac{FJ}~\cite{igarashi_featherweight_2001}.
 and \ac{FFJ}~\cite{apel_feature_2008}
In \ac{FFJ}, as well as in \ac{FFJ+}, classes can be added and 
modified by the introduction of a new feature (as discussed in the previous section).
An existing class can be extended by a class refinement. A class refinement is declared like a class but
preceded by the keyword \texttt{refines}. For example, \texttt{refines class C@feat \{$\dots$\}} refers to a class refinement that
refines the class \texttt{C}. The same can be achieved for method introduction and modification. Methods refinement,
however, will override the previous definition.

A syntactical difference between \ac{FFJ} and \ac{FFJ+} is that, in \ac{FFJ+}, 
the feature notion appears in the abstract syntax tree (AST) of the language.
While the designers of \ac{FFJ} argue that the programmer does not have 
to explicitly state which feature a class or method belongs to, 
we favored the approach of stating the feature in the name of every refinement.
This greatly simplifies the structure of the formalism of the language and can be 
seen as an information gathered by the parser to build the AST, and thus 
the actual code expressed using the concrete syntax of this language 
might not have these annotations.

In addition, an \ac{FFJ+} program has a table with every class 
declaration (\textsf{CT}) and another table with every class refinement (\textsf{RT}).
We make this distinction to simplify the extension from \ac{FJ} in \texttt{Coq}, since 
with this decision we eliminate the need to match whether a class in the table 
is a refination or a declaration. From this \textsf{RT} we can retrieve the composition order
of the refinements and build the refinement chain of the program, 
which is used to check if features were composed correctly and
does not references features that have not been introduced yet. 
Notice that we redefine the denotation of \textsf{RT} from \ac{FFJ}.
In the original version, it was used to retrieve the refinement name given a 
refinement declaration. This is no longer necessary in \ac{FFJ+}, since
that information is already encoded in the syntax.

Finally, in the original definition of \ac{FFJ}, the lookup functions are 
somewhat circumvoluted. Accordingly, we propose a very different approach
for them, with the aim as been not only as formal and simple as possible, 
but also easy to evolve from our mechanized version of \ac{FJ}. 
To this end, we eliminate the need for reverse field lookup, reverse method lookup, 
and the refinement relation. A formal description with all these changes 
is given in Section~\ref{subsec:lookup}. Note that, we were only 
able to conceive these improvements while formalizing \ac{FFJ+} in 
\texttt{Coq}. 


