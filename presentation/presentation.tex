\documentclass{beamer}

\usetheme{Madrid}
\usepackage[utf8]{inputenc}
\usepackage[brazil, american]{babel}

\title[Formalização de FFJ em Coq]{Formalização de Feature Featherweight Java \\ Utilizando Coq}

\author{Pedro da Costa Abreu Júnior}

\institute[UnB]
{
	Departamento de Ciência da Computação\\
	Universidade de Brasília 
}

\date{13 de Junho de 2016}

\subject{Formalização de Linguagens de Programação}


% Let's get started
\begin{document}
	
	\begin{frame}
		\titlepage
	\end{frame}
	
	\begin{frame}{Roteiro}
		\tableofcontents
	\end{frame}
	
	\section{Revisão Teórica}
	\begin{frame}{Revisão Teórica}{Linhas de Produto de Software}
		\centering
		Falar um pouco de SPLs \\
		blablabla
	\end{frame}
	
	\begin{frame}{Revisão Teórica}{Assistentes de Prova}
		\centering
		Falar um pouco dos assistentes de prova e do coq \\
		blablabla
	\end{frame}

	\section{Motivação}
	\begin{frame}{Motivação}
			\centering
			Formalizar Haephestus
	\end{frame}
	
	\section{Problema e Hipótese}
	\begin{frame}{Hipótese}
		\begin{block}{}
			Formalizar FFJ nos dará uma base sólida para compreendermos os próximos passos para formalizar um SPL
		\end{block}
	\end{frame}
			
	\begin{frame}{Problema}
		\begin{block}{Coq}
			Formalizar FFJ em Coq
		\end{block}
	\end{frame}
	
	\section{Featherweight Java}
	\begin{frame}{Featherweight Java}
		\centering
		Explicação de FJ, falar que não tem variable assignment e o que ten e tals
	\end{frame}
	\begin{frame}{Featherweight Java}{Exemplo de Programa}
		\centering
		Exemplo de um programa FJ
	\end{frame}
	\begin{frame}{Feature Featherweight Java}
		\centering
		Definição formal de FFJ falar qual a sacada e pans
	\end{frame}
	\begin{frame}{Feature Featherweight Java}{Exemplo de Programa}
		\centering
		Definição formal de FFJ
	\end{frame}
	

	
	\section{Justificativa e Motivação}
	
	
	\begin{frame}{Justificativa}
		Vamos aprofundar nossos conhecimentos de:
		\begin{itemize}
			\item Interpretadores
			\item Coq
			\item Linguagens de Programação
			\item Formalização de Linguagens de Programação
		\end{itemize}    
		\begin{block}{}
			Compreender próximos passos necessários para formalizar Linhas de Produtos de Software de modo geral
		\end{block}
	\end{frame}

	
	\section{Metodologia}
	\begin{frame}{Metodologia}
		\centering
		\begin{block}{}
		\textit{``You'll finally really learn whatever programming language you're writing a compiler for. There's no other way. Sorry!''} -  Steve Yegge
	\end{block}
	\end{frame}
	
	
	\section{Cronograma}
	\begin{frame}{Cronograma}
		\begin{table}
			\centering
			\caption{Cronograma para Segundo Semestre de 2016}
			\begin{tabular}{|r|c|c|c|c|c|c|c|}
				\hline						    &Jun&Jul&Ago&Set&Out&Nov& Dez \\ 
				\hline Formalizar FJ (Coq)   	&X	& X & X &   &   &   &  \\ 
				\hline Compreender FFJ			&	&	& X & X &   &   &  \\ 
				\hline Interpretador FFJ (Haskell)&	&	&   & X &   &   &  \\ 
				\hline Formalizar FFJ em Coq	&	&	&   & X & X & X & X\\ 
				\hline Escrever Artigo 			&   &   &   &   &   &   & X\\ 
				\hline 
			\end{tabular} 
		\end{table}
	\end{frame}
	
	\section{Resultados Esperados}
	
	\begin{frame}{Resultados Esperados}
		\begin{itemize}
			\item Implementação 
		\end{itemize}
	\end{frame}
	
	
	
	\appendix
	\section<presentation>*{\appendixname}
	\subsection<presentation>*{Referências}
	
	\begin{frame}[allowframebreaks]
		\frametitle<presentation>{Referências}
		
		\begin{thebibliography}{10}
			
			\beamertemplatearticlebibitems
			% Start with overview books.
			
			\bibitem{IntroMovel}
			Salzburg Research Forschungsgesellschaft m.b.H
			\newblock Developing Semantic CMS Applications - The IKS Handbook
			
			
			\beamertemplatearticlebibitems
			% Followed by interesting articles. Keep the list short. 
			\bibitem{IntroMovel}
			Berners-Lee, Tim and Lassila, Ora and Hendler, James
			\newblock The semantic web
			\newblock  Scientific America, 2001
			
			
			
			\beamertemplatearticlebibitems
			% Followed by interesting articles. Keep the list short. 
			\bibitem{IntroMovel}
			Berners-Lee, Tim and Cailliau, Robert and Groff, Jean-François and Pollermann, Bernd
			\newblock World-wide web: the information universe
			
			
			
		\end{thebibliography}
		
	\end{frame}
	
	
	\section*{Perguntas}
	\begin{frame}{  }
		\Large   \centering   Perguntas?
	\end{frame}
\end{document}



