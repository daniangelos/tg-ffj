\documentclass{beamer}

\usetheme{Madrid}
\usepackage[utf8]{inputenc}
\usepackage[brazil, american]{babel}
\usepackage{graphicx}

\usepackage{amsmath}
\usepackage{amsthm}
\usepackage{amssymb}
\usepackage{mathpartir}
\usepackage{adjustbox}


\title[Formalização de FFJ em Coq]{Formalização de Feature Featherweight Java \\ Utilizando Coq}

\author{Pedro da Costa Abreu Júnior}

\institute[UnB]
{
	Departamento de Ciência da Computação\\
	Universidade de Brasília 
}

\date{13 de Junho de 2016}

\subject{Formalização de Linguagens de Programação}


% Let's get started
\begin{document}
	
	\begin{frame}
		\titlepage
	\end{frame}
	
	\begin{frame}{Roteiro}
		\tableofcontents
	\end{frame}
	
	\begin{frame}{Revisão Teórica}{Featherweight Java}
		\centering
		\begin{block}{Featherweight Java}
			O que é FJ e FFJ?
		\end{block}
	\end{frame}
	
	\begin{frame}{Revisão Teórica}{Featherweight Java}
		\centering
		\begin{itemize}
			\item {É uma linguagem feito para modelar um núcleo mínimo do Java}
			\item {Proposto por teóricos renomados, entre eles \textbf{B. Peirce}}
			\item Facilmente extensível
			\item Vem sendo utilizado com sucesso	
		\end{itemize}
	\end{frame}
	
	\begin{frame}{Revisão Teórica}{Featherweight Java}
		\centering
		\begin{figure}[h!]
			\includegraphics[width=\textwidth]{"FJ citations".png}{}
			\caption[Quantidade de citações do Paper de FJ; This is the second line]
			{\tabular[t]{@{}l@{}}
				Citações do Paper de FJ \\ 
				\scalebox{.5}{\small{Imagem retirada do Google}}\endtabular}
			
			\label{fig:fj-citations}
		\end{figure}
	\end{frame}
	
	\begin{frame}{Revisão Teórica}{Características de FJ}
		FJ possui as seguintes características:
		\begin{itemize}
			\item Classes;
			\item Métodos;
			\item Atributos (fields);
			\item Herança;
			\item Typecasts.
		\end{itemize}
	\end{frame}
	
	\begin{frame}{Revisão Teórica}{Características de FJ}
		FJ \textbf{omite} as seguintes características:
		\begin{itemize}
			\item Assignment;
			\item Interfaces;
			\item Overload;
			\item Null;
			\item Controle de acesso (p.ex. public e private);
			\item Metodos abstratos.
		\end{itemize}
	 \end{frame}
	
	\begin{frame}{Revisão Teórica}{Featherweight Java}
		\centering
		\begin{table}[ht!]
			\caption{Abstract Syntax}
			\begin{tabular}{ccl}
				$L$&~::= & $class\ C~extends~C\ \{\bar{C} \ \bar{f};\ K\
				\bar{M}\}$\\ 
				\vspace{0.8mm}
				$K$&~::= &
				$C~(\bar{C}~\bar{f})\
				\{super~(\bar{f});~this.\bar{f}=\bar{f};\}$\\
				\vspace{0.8mm}
				$M$&~::= & $C~m~(\bar{C}~\bar{x})\ \{return~e;\}$\\
				\vspace{0.8mm}
				$e$&~::= & $x~|~e.f~|~e.m~(\bar{e})~|~new~C~(\bar{e})~|~(C)~e$ \\
			\end{tabular} \\
			\vspace{1.5mm}
			\label{abstractsyntax}
		\end{table}
	\end{frame}
	
	\begin{frame}{Revisão Teórica}{Feature Featherweight Java}
		\centering
		\begin{block}{Feature Featherweight Java}
			Adiciona o conceito de features a FJ
		\end{block}
	\end{frame}

	
	\begin{frame}{Revisão Teórica}{Assitentes de Prova}
		\centering
		\begin{block}{Assistentes de Prova}
			Erros em provas formais podem passar despercebidos a humanos.\\
			Mas não para computadores!
		\end{block}
		
		\begin{block}{Coq}
			\begin{itemize}
				\item Possui um núcleo checado mecanicamente;
				\item Possui uma boa documentação e livros didáticos;
				\item Possuo um prática razoável.
			\end{itemize}
		\end{block}
	\end{frame}

	
	\section{Problema e Hipótese}
	\begin{frame}{Problema}
		\begin{block}{}
			Ainda não existe checagem assistida por computadores de FFJ. \\
			Isto indica que falhas possam existir nas provas de corretude da linguagem.
		\end{block}
	\end{frame}
	
	
	\begin{frame}{Hipótese}
		\begin{block}{}
			Será possível utilizar o livro Software Foundations para alavancar o trabalho inicial no Coq e seguir a partir daí sem maiores dificuldades.
		\end{block}
	\end{frame}
			
	\section{Justificativa}
	
	
	\begin{frame}{Justificativa}
		Vamos aprofundar nossos conhecimentos de:
		\begin{itemize}
			\item Interpretadores
			\item Coq
			\item Linguagens de Programação
			\item Formalização de Linguagens de Programação
		\end{itemize}    
		\begin{block}{}
			Compreender próximos passos necessários para formalizar Linhas de Produtos de Software de modo geral
		\end{block}
	\end{frame}

	
	\section{Metodologia}
	\begin{frame}{Metodologia}
		\centering
		\begin{block}{}
		\textit{``You'll finally really learn whatever programming language you're writing a compiler for. There's no other way. Sorry!''} -  Steve Yegge
	\end{block}
	\end{frame}
	
	
	\section{Motivação}
	\begin{frame}{Motivação}
		\centering
		\scalebox{2}{Formalizar Haephestus}
	\end{frame}
	
	
	\section{Cronograma}
	\begin{frame}{Cronograma}
		\begin{table}
			\centering
			\caption{Cronograma para Segundo Semestre de 2016}
			\begin{tabular}{|r|c|c|c|c|c|c|c|}
				\hline						    &Jun&Jul&Ago&Set&Out&Nov& Dez \\ 
				\hline Formalizar FJ (Coq)   	&X	& X & X &   &   &   &  \\ 
				\hline Compreender FFJ			&	&	& X & X &   &   &  \\ 
				\hline Interpretador FFJ (Haskell)&	&	&   & X &   &   &  \\ 
				\hline Formalizar FFJ em Coq	&	&	&   & X & X & X & X\\ 
				\hline Escrever Artigo 			&   &   &   &   &   &   & X\\ 
				\hline 
			\end{tabular} 
		\end{table}
	\end{frame}
	
	\section{Resultados Esperados}
	
	
	\appendix
	\section<presentation>*{\appendixname}
	\subsection<presentation>*{Referências}
	
	\begin{frame}[allowframebreaks]
		\frametitle<presentation>{Referências}
		
		\begin{thebibliography}{10}
			
			\beamertemplatearticlebibitems
			% Start with overview books.
			
			\bibitem{IntroMovel}
			Harper, Robert. 
			\newblock Practical foundations for programming languages. 
			\newblock Cambridge University Press, 2012.
			
			
			\beamertemplatearticlebibitems
			% Followed by interesting articles. Keep the list short. 
			\bibitem{IntroMovel}
			Igarashi, Atsushi, Benjamin C. Pierce, and Philip Wadler.
			\newblock Featherweight Java: a minimal core calculus for Java and GJ.
			\newblock ACM Transactions on Programming Languages and Systems (TOPLAS) 2001.
			
			
			
			\beamertemplatearticlebibitems
			% Followed by interesting articles. Keep the list short. 
			\bibitem{IntroMovel}
			Apel, Sven, Christian Kästner, and Christian Lengauer.
			\newblock Feature Featherweight Java: A calculus for feature-oriented programming and stepwise refinement
			\newblock Proceedings of the 7th international conference on Generative programming and component engineering. ACM, 2008.
			
			
			
		\end{thebibliography}
		
	\end{frame}
	
	
	\section*{Perguntas}
	\begin{frame}{  }
		\Large   \centering   Perguntas?
	\end{frame}
	
	\begin{frame}{Regras de Tipagem FJ}
		\begin{table}[h!]
			\centering
			\def\arraystretch{3}
			\adjustbox{max height=\dimexpr\textheight-5.5cm\relax, max width=\textwidth}{
			\begin{tabular}{cr}
				$\Gamma \vdash x:\Gamma(x)$& (T-Var)\\
				
				\inferrule{\Gamma \vdash e_{0}:C_{0}\qquad fields~(C_{0})=\bar{C}\
					\bar{f}}
				{\Gamma \vdash e_{0}.f_{i}:C_{i}} & (T-Field)\\
				
				\inferrule{\Gamma \vdash e_{0}:C_{0}\qquad
					mtypes~(m,~C_{0})=\bar{D}\rightarrow C\qquad \Gamma \vdash
					\bar{e} : \bar{C} \qquad \bar{C}~<:~\bar{D}}
				{\Gamma \vdash e_{0}.m(\bar{e}):C} & (T-Invk)\\
				
				\inferrule{fields(C)=\bar{D}\ \bar{f}\qquad \Gamma \vdash
					\bar{e}:\bar{C} \qquad \bar{C}~<:~\bar{D}}
				{\Gamma \vdash new\ C(\bar{e}):C} & (T-New)\\
				
				\inferrule{\Gamma \vdash e_{0}:D \qquad D~<:~C}
				{\Gamma \vdash (C)~e_{0}: C} & (T-UCast)\\
				
				\inferrule{\Gamma \vdash e_{0}:D\qquad C~<:~D \qquad C \neq D}
				{\Gamma \vdash (C)~e_{0}:C} & (T-DCast)\\
				
				\inferrule{\Gamma \vdash e_{0}:D\qquad C~\nless :~D \qquad D~\nless:~C 
					\qquad stupid\ warning}
				{\Gamma \vdash (C)~e_0:C} & (T-SCast)\\
				
			\end{tabular}}
			\vspace{1.5mm}
			\label{exptyping}
		\end{table}
	\end{frame}
	
	\begin{frame}{Regras de Computação FJ}
		\begin{table}[h!]
			\centering
			\def\arraystretch{3}
			\begin{tabular}{cr}
				\inferrule{fields~(C) = \bar{C} \bar{f}}
				{(new\ C(\bar{e})).f_i \rightarrow e_i} & (R-Field) \\
				
				\inferrule{mbody~(m, C) = \bar{x}.e_0}
				{(new\ C~(\bar{e})).m~(\bar{d}) \rightarrow[\bar{d}/\bar{x}, new\ C~(\bar{e})/this]e_0} & (R-Invk)\\
				\inferrule{C<:D}
				{(D)(new\ C~(\bar{e})) \rightarrow new\ C~(\bar{e})} & (R-Cast)\\
			\end{tabular}
			\vspace{1.5mm}
			\label{expcomput}
		\end{table}
	\end{frame}
	
	
	\begin{frame}{Regras de Congruencia FJ}
		\begin{table}[h!]
			\centering
			\def\arraystretch{3}
			\begin{tabular}{cr}
				\inferrule{e_0 \rightarrow e_0'}
				{e_0.f\rightarrow e_0'.f} & (RC-Field) \\
				\inferrule{e_0 \rightarrow e_0'}
				{e_0.m~(\bar{e})\rightarrow e_0'.m~(\bar{e})} & (RC-Invk-Recv) \\
				\inferrule{e_i \rightarrow e_i'}
				{e_0.m~(\dots, e_i, \dots) \rightarrow e_0'.m~(\dots, e_i, \dots)} & (RC-Invk-Arg) \\
				\inferrule{e_i \rightarrow e_i'}
				{new\ C~(\dots, e_i, \dots) \rightarrow new\ C~(\dots, e_i', \dots)} & (RC-New-Arg) \\
				\inferrule{e_0 \rightarrow e_0'}
				{(C)e_0 \rightarrow (C)e_0'} & (RC-Cast) \\
				
			\end{tabular}
			\vspace{1.5mm}
			\label{expcongr}
		\end{table}
	\end{frame}
	
	\begin{frame}{Subtipagem FJ}
		\begin{table}[ht!]
			\centering
			\label{subtyping}
			{\renewcommand{\arraystretch}{3}
			\begin{tabular}{c}
				$C~<:~C$ \\
				\inferrule{C <: D \qquad C <: E}
				{C <: E} \\
				\inferrule{class~C~extends~D~\{~\ldots~\}}
				{C~<:~D}
			\end{tabular}}
			\vspace{1.5mm}
		\end{table}
	\end{frame}
\end{document}



